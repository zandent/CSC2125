\documentclass{article}
\usepackage[utf8]{inputenc}
\usepackage{amsmath,amssymb}
\usepackage{makecell}
\usepackage[a4paper, total={6in, 9in}]{geometry}
\usepackage{authblk}
\usepackage{graphicx}
\graphicspath{{./figures/}}
\usepackage{sectsty}
\sectionfont{\fontsize{11}{15}\selectfont}
\author{Zi Han Zhao}
\affil{1001103708}
\date{}
\title{CSC2125 Homework 3}
\begin{document}
\maketitle
\renewcommand{\thesubsection}{(\alph{subsection})}
\section{Suppose we run PBFT algorithm on a cluster of 21 machines. What is the
maximum number of machines we will be able to tolerant for failure
simultaneously?}
We know $N\geqslant3f+1$ where N is the total number of nodes and f replicas are faulty. Then $21\geqslant3f+1$ so $f_{max}=6$.
\section{A node in PBFT waits for prepare/commit messages from two thirds of nodes
(replicas) during the prepare/commit stages. What would happen if the node
instead waits for prepare and commit messages from only a simple majority of
other nodes? Would the modified PBFT be secure? If not, please explain with a
counter example.}
%If a node doesn't wait at least $2f+1$ replicas responses,
The modified PBFT is not secure anymore.
Let's say the number of responses of node A is x where $x<2f+1$, 
$f$ is the number of faulty nodes. Among the x messages, 
in the worst case, 
there would be possibly $f$ faulty responses so there are $x-f$ non-faulty responses. 
It is found that $x-f<f+1$, 
which means that non-faulty responses is not greater than faulty responses. 
Therefore, node A cannot decide the message he received is correct or not.
\section{PBFT relies on the primary node to send out pre-prepare messages to drive the
consensus process. What would happen if the primary node is malicious or fails?
How does PBFT handle this situation?}

\end{document}